\documentclass[11pt]{article}

\usepackage{fullpage}
\usepackage{xspace}

\newcommand{\NumRacks}{\ensuremath{\mathit{N}}} 
\newcommand{\NumFSOs}{\ensuremath{\mathit{M}}} 
\newcommand{\Rack}{\ensuremath{\mathit{Rack}\xspace}} 
\newcommand{\FSO}{\ensuremath{\mathit{FSO}\xspace}} 
\newcommand{\RackIndex}{\ensuremath{\mathit{i}}} 
\newcommand{\FSOIndex}{\ensuremath{\mathit{j}}} 

\newcommand{\RackFSOa}{\ensuremath{\RackIndex,\FSOIndex}} 
\newcommand{\RackFSOb}{\ensuremath{\RackIndex',\FSOIndex'}} 

\newcommand{\RackFSOc}{\ensuremath{\RackIndex,\FSOIndex}} 
\newcommand{\RackFSOd}{\ensuremath{\RackIndex,\FSOIndex}} 



\newcommand{\Flow}{\ensuremath{\mathit{f}}} 
\newcommand{\InterRack}{\ensuremath{\mathit{k}}}
\newcommand{\Traffic}{\ensuremath{\mathit{T}}}
\newcommand{\Edge}{\ensuremath{ (\RackFSOa \rightarrow \RackFSOb) }} 
\newcommand{\EdgeRev}{\ensuremath{ (\RackFSOb \rightarrow \RackFSOa) }} 
\newcommand{\Edgefoo}{\ensuremath{ (\RackFSOc \rightarrow \RackFSOd) }} 
\newcommand{\EdgeVar}{\ensuremath{\mathit{d}}}
\newcommand{\EdgeVarIndex}{{\EdgeVar_{\Edge}}}
\newcommand{\EdgeVarRevIndex}{{\EdgeVar_{\EdgeRev}}}
\newcommand{\Capacity}{\mathit{C}}
\newcommand{\FlowEdge}{\ensuremath{\Flow_{\InterRack,\Edge}}} 




\title{Quick and dirty optimal topology formulation}

\begin{document}

\maketitle

\medskip\noindent {\bf Preliminaries:}  We consider a setting with $\NumRacks$ 
 {\Rack}s each equipped with $\NumFSOs$ {\FSO} devices. 
 We use the index $\RackIndex$ to refer to a specific {\Rack} 
 and  $\FSOIndex$ to refer to a specific {\FSO} device. 
  Thus, the subscript $\RackFSOa$  will refer to a specific 
 Rack-FSO combination.


 Let $\InterRack$ denote a pair of racks and let us assume that we know some estimated traffic matrix for every pair 
 of racks denoted by $\Traffic_\InterRack$ for the volume of 
 traffic from $\InterRack$.

Suppose the bandwidth of an inter-FSO link (if enabled) is $\Capacity$. 

 
\medskip\noindent{\bf Control variables:}
Our goal is to decide which pair of Rack-FSO links that need to be connected -- 
 let $\EdgeVarIndex$ be a binary variable that takes the value 
 1 if we choose to add a link between $\Edge$. Note that we are considering 
 a fully steerable model where every pair of FSO trasceivers 
 can talk to each other.   


\medskip\noindent{\bf Formulation:}

 Let $\FlowEdge$ denote the volume of the inter-rack traffic on $\InterRack$ 
 that flows on the edge $\Edge$.  

Then, we have a natural sanity check that the flow can only be positive and non-zero 
 only if we have an edge enabled.

\begin{equation}
 \forall \FlowEdge:   0 \leq \FlowEdge \leq \Capacity \times \EdgeVarIndex 
\label{eq:edgesanity}
\end{equation}

We have a flow conservation constraint on a per-rack basis for rack that is not the ingress or egress: 

\begin{equation}
 \forall \InterRack, \forall \RackIndex: \sum_{ \RackIndex', \FSOIndex', \FSOIndex}       \Flow_{\InterRack, (\RackIndex', \FSOIndex' \rightarrow \RackIndex, \FSOIndex)  }    = \sum_{ \RackIndex', \FSOIndex', \FSOIndex} \Flow_{ \InterRack, (\RackIndex, \FSOIndex \rightarrow \RackIndex', \FSOIndex')  }  
\label{eq:flowcons}
\end{equation}


For the ingress, we have a special constraint: 

\begin{equation}
 \forall \InterRack, s.t.\ \RackIndex = \mathit{Ingress}(\InterRack):   \sum_{ \RackIndex', \FSOIndex', \FSOIndex}       \Flow_{\InterRack, (\RackIndex, \FSOIndex \rightarrow \RackIndex', \FSOIndex')   }    = \Traffic_\InterRack 
\label{eq:flowconsin}
\end{equation}

For the egress, we have a special constraint: 

\begin{equation}
 \forall \InterRack, s.t.\ \RackIndex = \mathit{Egress}(\InterRack):   \sum_{ \RackIndex', \FSOIndex', \FSOIndex}       \Flow_{ \InterRack, (\RackIndex', \FSOIndex' \rightarrow \RackIndex, \FSOIndex )  }    = \Traffic_\InterRack 
\label{eq:flowconsin}
\end{equation}



Each FSO on each rack can be connected to at most one other receiver.
\begin{equation}
 \forall \RackFSOa:  \sum_{\RackFSOb} \EdgeVarIndex \leq 1  
\label{eq:degreecons}
\end{equation}



By symmetry, each FSO on each rack can be connected to at most one other sender. 
\begin{equation}
 \forall \RackFSOb:  \sum_{\RackFSOa} \EdgeVarIndex \leq 1  
\label{eq:degreecons}
\end{equation}


We need to model  the bidirectionality of the links as well: 
By symmetry, each FSO on each rack can be connected to at most one other sender. 
\begin{equation}
 \forall \RackFSOa:  \EdgeVarIndex =  \EdgeVarRevIndex
\label{eq:degreecons}
\end{equation}




\medskip\noindent {\bf Objective:}  Our goal is to accomodate the demands represented
by our traffic matrix in the most efficient manner.  In other words, we want to
minimize the congestion on the maximum loaded inter-FSO link  so  that the
completion time for the maximum flow is the smallest. 


\begin{equation}
 \mathit{Minimize}  \max_{\Edge}  \frac{\sum_{\InterRack}\FlowEdge}{\Capacity }
\end{equation}



\end{document}

