\section{Discussion} 
\label{sec:future}

%Further Challenges}

%\green{Before we conclude, we discuss other research challenges and opportunities
% that  our  vision of a flexible FSO-based datacenter network  introduces. }

\para{Rethinking Metrics of Goodness.}  Traditional  metrics  such as  bisection bandwidth and diameter largely reflect a
{\em static} perspective of the topology. For the types of flexible networks we
envision, we need to  rethink these metrics; e.g., 
we  need a notion of  {\em dynamic} bisection bandwidth  based on the best achievable
  bandwidth by some {\em realizable} topology for a given 
network partition.

\para{Optimal Topologies.} Given new dynamic performance indices, we need to
reason about the  pre-configured alignment of SMs that optimizes these
metrics. While the random and extended hypercube designs work well, we do not
know if these are provably (near-)optimal.  Furthermore, choosing an optimal
run-time topology is effectively an  online optimization problem---given a
pre-configured topology, current configuration and traffic patterns, what is
the best way to reconfigure the network?  What makes this  challenging is that
even the offline version of this problem is intractable. 
 
%Here, the switching latency should be taken into consideration. We note that
%even the {\em offline} version of the problem is computationally intractable.

\para{FSOs for Modularized Data centers.} While our current work
focuses on the inter-rack fabric, FSOs might also be useful for
 containerized architectures~\cite{helios}.  This context
introduces new challenges and opportunities.  Specifically, a ceiling
mirror is not feasible in outdoor scenarios and we need other
mechanisms (e.g., vertically steerable FSOs?) for line-of-sight. At
the same time, the coarser aggregation may permit higher switching
 latencies and thus be amenable to slower (mechanical) steering
mechanisms that can provide full reconfigurability.

\eat{
This could imply that we may be less concerned by the cost of
individual FSOs (as there are fewer).  Furthermore, the traffic
patterns may be more predictable, and thus amenable to other steering
technologies (see below).}

\para{Multipath and Traffic Engineering.}  We could further improve
the performance using multi-path TCP~\cite{multipathtcp} or better traffic
engineering~\cite{thorup}. We posit that multi-path TCP has 
natural synergies with reconfigurability as it can  
alleviate transient congestion and connectivity issues.


%how does topology  design affect these?

%Our use of SMs with a pre-configured
%topology represents a practical FSO architecture that can be
%cost-effectively implemented given today's commercially available
%technologies. That said,



%\para{More Degrees of Freedom.}  Future FSO designs may provide more degrees of
%freedom: mechanical and non-mechanical steering of SMs and  lasers~\cite{elec-survey}.  An 
% interesting question is to  match the
%timescales of reconfiguration each mechanism offers vs.\ the predictability 
%of traffic patterns; e.g., planned  VM migrations 
% may be amenable to slower steering options.

% and the
%  dynamicity of the traffic.} \red{Can be made clearer.}

\para{Other Benefits.} In addition to the quantitative benefits we
explored, FSO-based flexible architectures also offer other qualitative
advantages. First, by acting as an enabler for new topologies, it naturally
inherits the  properties  they provide; e.g., random graphs offer incremental
expandability~\cite{jellyfish}. Second, selectively disabling links may also
decrease energy costs~\cite{elastictree-nsdi10}.  Furthermore, by eliminating
the wired infrastructure, FSOs can potentially reduce  cooling costs by 
 avoiding  problems due to airflow  obstruction~\cite{airflow}.

%and power (e.g.,  by eliminating aggregation switches) costs.
 
%  VS: dont know what navid had .. but here are a couple ones:
	% http://www.cisco.com/en/US/solutions/collateral/ns340/ns517/ns224/ns783/white_paper_c11-473501.html 
	%http://www.mm4m.net/library/Avoidable_Mistakes_that_Compromise_Cooling_Perfomance.pdf

%  power/cooling, cabling cost 


%ese future designs will require novel
%techniques for optimal topology design.

%\vyas{Exploring nonmechanical steering platforms goes here}

