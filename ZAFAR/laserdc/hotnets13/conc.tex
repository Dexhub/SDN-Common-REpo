\section{Conclusions}
\label{sec:conc}
We explored an FSO-based inter-rack fabric for data centers, a solution whose
benefits have been suggested~\cite{us-patent,hotnets09scribe}, but has
received little attention in depth. We showed that FSOs can be viable with the
extensions we propose (e.g., switchable mirrors and pre-configured
topologies).  Our evaluations show that FSO-based designs offer good
cost vs.\ performance tradeoffs (Table~\ref{table:cost}) w.r.t.\ state-of-art
solutions; e.g., close to 9~Gbps bisection bandwidth at much less cost
compared to a Fat-tree, and 90\% of the performance for 2~Gbps fat-tree at
70\% of the cost.  We note that these benefits only represent an early
starting point---miniaturization and commoditization will further improve
the cost-performance tradeoffs and flexibility that FSO-based designs can
offer.



%further remove many of the
%cost/size constraints and enable \blue{greater flexibility} for
%FSO-based datacenters.

\eat{We are currently exploring several dimensions to turn this
  initial promise into reality: provably optimal topologies, scalable
  and near-optimal reconfiguration algorithms, and demonstrating a
  small-form factor prototype.}



%\blue{In
%  essence, our work provides only a glimpse of what could be achieved
 % with an FSO-based design in DC. Miniaturization and lowered-costs of
%  FSO devices with scale could remove some of the limitations, and
%  make the overall solution very compelling.} 
 
  

 % 70\% better
%Furthermore, it  acts as
%an enabler to leverage the benefits of topology designs that might be
%intractable due to wiring concerns.
%throughput for 50\% more cost for a 1~Gbps fat-tree;
