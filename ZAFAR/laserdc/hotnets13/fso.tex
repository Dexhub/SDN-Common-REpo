\section{FSO: Motivation and Viability} 
\label{sec:fso}

%Our overarching vision is to explore an all ``wireless'' inter-rack
%fabric that can offer several natural advantages over traditional
%wired architectures: it obviates the need for a fixed infrastructure,
%it does not need to be overprovisioned for the ``worst-case'', it can
%be easily adapted depending on future demands, and it offers better
%incremental expandability.

In this section, we provide background on free-space optics
(FSO) and argue why this technology can serve as the ``workhorse''  for our goals.

\subsection{Background on FSO} 

Free-space optical (FSO) communication~\cite{kedar} uses modulated
visible or infrared (IR) laser beams in the free space to implement a
communication link. Unlike traditional optical networks, the laser
beam in FSO is not enclosed in a glass fiber, but transmitted through
the air. There are \blue{two main} benefits of FSO compared to
traditional RF technologies (e.g., WiFi or 60~GHz) that make it a
promising candidate for \DCs:

\para{(1) Low Interference and Bit-Error Rates.} FSO uses very narrow laser beam
widths, \cam{diverging with an angle of a} few milliradians or less (1
milliradian = 0.0573 degree). This reduces the interference footprint to a
negligible level.  Thus, unlike traditional RF technologies, FSO communications
from multiple senders do not interfere, unless the senders are aligned to the
same receiver. Minimal interference and narrowness of the beams also results in
very low bit-error rate.

\para{(2) High Bandwidth over Long Ranges.}  Optical communications inherently
provide significantly higher bandwidth than  current RF technologies owing to
the use of much higher frequency and absence of regulatory
restrictions~\cite{kedar}. Coupled with much lower attenuation of power over
distance, FSO links are able to offer higher data rates at long distances
(several kms) even with modest transmit power
(watts)~\cite{kedar}; e.g., commercially available FSO devices 
 provide 2.5Gbps~\cite{fsona}, and demonstration systems even report 
 few Tbps~\cite{mustafa2013reintroducing}.

\subsection{Feasibility of FSO in the Data center} 

%The use of lasers naturally requires perfect line-of-sight. Thus, 
The main stumbling block for traditional FSO communication comes from
atmospheric elements (e.g., rain, fog, dust) and background radiation (e.g.,
sunlight).  In the indoor and controlled environment of a \DC, these
challenges largely disappear. 
%
However, key challenges remain.  First, commercially available FSO systems are
bulky, expensive (\$5-10K for a single link), and power hungry.  
%
Second, FSO beams require a clear line-of-sight, and thus, obstacle avoidance
is a potential issue. (We defer the issue of beam alignment to the next
section.)

\para{Cost, Size and Power:} Today's commercial FSO devices are
relatively bulky $\approx$ 2 cubic feet of volume
(e.g.,~\cite{lightpointe}).
% e.g, an off-the-shelf we found was $12.6"\times 11.7"\times
%  24.4$~\cite{lightpointe}.  
\blue{This stems from many factors mandated by outdoor use-cases, viz., 
use of multiple laser beams to provide spatial diversity, elaborate
alignment mechanisms needed for long distance use and recovery from
building swaying, ruggedization needs etc.}
%
In contrast, a \DC-centric FSO
  device can be conceptually built by repurposing  
  commonly used optical small form-factor
  pluggable (SFP) transceivers~\cite{sfp}.  %SFP is a standards-based compact
%  transceiver to interface a standard network device (e.g., switch,
% router, media converter or similar device) to a fiber optic
 % cable. 
 The real difference between an \cam{optical} SFP transceiver and an
  FSO device is that the former interfaces directly with an optical fiber instead
  of transmitting the laser signal through the air. 
\cam{Converting optical SFP to FSO entails the  use of 
collimating lenses on the optical path and an alignment 
mechanism (e.g., precision positioners with a camera), though
  of a lesser level of complexity than that needed for alignment for
  outdoors and very long distances. }
  
  Several projects have demonstrated the viability of this approach without
extra
amplification~\cite{tsujimura2008trans,mustafa2013reintroducing,yoshida2013focus},
including one that uses commodity components~\cite{mustafa2013reintroducing} and one that targets Tbps speeds between
buildings~\cite{ciaramella2009terabit}.
%  Thus, the only additional components needed to build an FSO would be a
%  focusing lens, a small camera (for alignment), and the alignment machinery. 
  Given that 10Gbps SFP transceivers  cost about US\$250~\cite{sfp}, we
estimate that an FSO device can be built for roughly \$750. 
  
  With respect to  size, SFPs themselves are small.  After reviewing the basic
design requirements of the mirror and alignment mechanisms, we believe that the
entire assembly can be put together within about $\approx$ \mbox{3'' x 8''} 2D
footprint \blue{that could provide a usable range of
100-200m~\cite{mustafa2013reintroducing,ciaramella2009terabit}. This range
would normally cover the needs of most \DCs.} Finally, in terms of power, we
note that with no additional amplification needed, the bulk of the power will
be consumed in  the SFP component, which is $\leq$ 1~watt. 
%\it{maximum}.
  
%  Since optical SFPs themselves are small (\mbox{2'' x 1''}), the FSO device
%  described above would be $\approx$ \mbox{3'' x 8''}.

% for an FSO device equipped with five
%\mbox{1'' x 1''} switchable mirrors (discussed later in \S\ref{sec:design}).}

\para{Circumventing Physical Obstructions:} If we have multiple FSO devices
on the top of each rack, then the devices are likely to be obstacles for other 
 links. We avoid this by   leveraging 
ceiling mirrors~\cite{3db}. 
%\blue{(Note, however, that ceiling mirrors serves a slightly
%different purpose in our context. In 60~Ghz communication, the ceiling mirrors
%reduce interference. In contrast, we use it to establish clear line of sight
%between arbitrary pairs of transceivers.)} \red{SD: suggest remove.} 
Specifically, we avoid obstructions by  directing FSO beams  upwards and
reflecting them  from the ceiling mirrors  (Figure~\ref{fig:link}).
Conventional mirrors themselves can easily reflect visible and IR FSO beams with negligible
loss (see~\S\ref{sec:design-sm}) and thus, the cost of a ceiling mirror is negligible.
 

%\end{packeditemize}


%\red{This is not the right place. SMs haven't been introduced yet.} 
 %% Agree ..

%Having established the viability of FSO in the \DC, next we describe
%our overall architecture for realizing an FSO-based inter-rack fabric. 

%VS: dropping .. added the pointer earlier \blue{An important challenge we will address there is that
%of beam alignment while imparting the ability of switching.}


%In summary, we find that FSO can be a potential ``enabler'' for an
%all-wireless datacenter fabric: it 
%provides a much better transmit power, range and data rate tradeoff with respect
%to RF as well as automatic immunity from interference from other links. 
%Some of its traditional problems (e.g., impact of atmosphere) largely go away in the datacenter context
%and other requirements (e.g., line of sight requirement) can be handled by a judicious design.

%On 
%the other hand
%
%significantly higher
%bandwidth, longer range, and lower interference compared to RF-based
%technologies.  Furthermore, many traditionally perceived challenges
%for FSO (e.g., attenuation, size, power, cost) will
%likely disappear in a datacenter context.


%\vyas{there is a meta issue missing here that the FSOs are on the top of each rack etc}

