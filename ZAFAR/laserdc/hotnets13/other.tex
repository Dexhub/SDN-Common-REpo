\section{Introduction}

Data centers (DCs) are a critical piece of today's computing infrastructure that
support key Internet applications.  In this context, data center network designs
must satisfy several potentially conflicting goals---performance (e.g.,
minimize oversubscribed links, low latency)~\cite{fattree,vl2}, equipment and
management cost~\cite{fattree,popa-cost}, flexibility to adapt to changing
traffic patterns~\cite{proteus,3db,flyways,cthru}, incremental expandability to
add new servers or racks~\cite{legup,jellyfish}, and other practical concerns
including cabling complexity, power, and
cooling~\cite{farrington,portland,cabling}.

Given the highly variable and unpredictable nature of data center
workloads~\cite{vl2}, early data center designs offered extreme points
in the space of cost-performance tradeoffs---either poor performance
at low cost (e.g., a simple tree has many oversubscribed links) or
expensive over-provisioned solutions with good worst-case performance
(e.g., fat-trees for full bisection bandwidth~\cite{fattree}). Recent
works suggest a ``middle ground'' that dynamically augments a simple
fixed infrastructure with additional inter-rack
wireless~\cite{flyways,3db} or optical links~\cite{cthru} to
alleviate congested hotspots.  While these do offer performance
benefits, we believe that these do not push the envelope far enough.

% early datacenter designers were initially 

%Early datacenter designs used structured topologies 
%such as  simple tree architectures or fat-tree style
%topologies~\cite{fattree}. These designs, unfortunately, cannot
%flexibly adapt to changing traffic patterns;  e.g., if some pairs of
%racks become hotspots~\cite{flyways,3db,vl2,cthru}.
% and they are not
%incrementally expandable (e.g., with a fat-tree one can only expand in
%discrete increments).  
% Consequently, datacenter operators need to
%compensate for this lack of flexibility  by
%  overprovisioning the network~\cite{fattree,jellyfish}.

%} \ref{Not %  true?}
 %}

%\footnote{While
%  optical interconnects have also been suggested~\cite{cthru}, the use
%  of wireless is more appealing as it does not burden operators with
%  cabling costs.} %\red{I would just move the footnote inline ..} 

In this work, we take the flexibility offered by wireless inter-rack
links to the logical extreme and explore the vision
of an \emph{all-wireless inter-rack data center fabric}.  This vision,
if realized, would provide unprecedented degrees of flexibility for
data centers.  For example, it will allow operators to dynamically
reconfigure the {\em entire} network topology to adapt to changing
traffic demands.  Similarly, it can act as an enabler for operators to
explore and benefit from topology structures  that would otherwise remain
``paper designs'' due to the perceived cabling complexity.

%(e.g., random graphs that offer better incremental expandability~\cite{jellyfish}),
%\vyas{something else}

Unfortunately, existing wireless/RF technologies are not suitable on
two fronts. First, they incur a large interference footprint even with
advanced phased-array antennas, especially when side lobes are
considered~\cite{cornell}. Second, they suffer from a significant
drop-off in data rates at longer distances~\cite{3db}, especially
since federal regulations prevent use of higher power or wider
bandwidth.  In conjunction, these factors fundamentally limit the
number and types of inter-rack links that can be created---an issue
that exists even with newer designs to extend the range via ceiling
reflectors~\cite{3db}.

Consequently, we look beyond traditional RF-based solutions and
explore a somewhat non-standard ``wireless'' technology, namely {\em
  Free-Space Optics} (FSO).  FSO uses visible or infra-red lasers to
implement point-to-point data links, at very high data rates
($>$1~Gbps) and at longer ranges. (We elaborate on these
in \Section\ref{sec:fso}). While the use of FSO in a general
communications context is not new, there has been very little work to
systematically explore the viability of FSO in the data center or
highlight the benefits that FSO offers in this context.\footnote{The
  only parallel work we are aware of is a recent patent
  document~\cite{us-patent}. Unfortunately, this offers little in
  terms of viability analysis, design space arguments, or
  performance tradeoffs.}


While FSO is an enabling technology, there are two significant
challenges that need to be addressed before we can realize a flexible
DC network fabric.  First, with off-the-shelf technology, FSO
  links cannot be cost-effectively reconfigured (i.e., realigned) at
  fast time-scales necessary in the DC context. Second, cost and
physical constraints limit the number of FSO links that can be
installed on the top of each rack.  Thus, we need an effective
topology design and management framework that can provide the desired
performance and flexibility while operating under the physical and
cost constraints.


In this paper, we discuss the viability of an FSO-based all-wireless inter-rack
fabric and present early solutions to address the above challenges.
Specifically, to achieve fast reconfigurability, we leverage ``switchable
mirrors'' that can be electronically controlled to act as mirrors or
pass-through glasses; the use of switchable mirrors effectively obviates the
need for careful (re)alignment (\S\ref{sec:design}). We also discuss topology
design and reconfiguration heuristics that seem to work well in practice.  We
have built a proof-of-concept prototype using off-the-shelf FSO and switchable
mirror solutions. We present simulation results that demonstrate superior
performance of our initial design compared to state-of-art data center network
designs of comparable cost (\Section\ref{sec:perf}). We discuss extensions 
 to our basic approach in \S\ref{sec:future}, before concluding in \S\ref{sec:conc}.
  We discuss related work inline through the paper.











