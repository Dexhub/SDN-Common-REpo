\documentclass[11pt]{article} 

\usepackage[usenames,dvipsnames]{color}
\usepackage{wrapfig}
\usepackage{times}
\usepackage{xspace}
\usepackage{subfig}
\usepackage{soul}
\setlength{\oddsidemargin}{0in}
\setlength{\evensidemargin}{0in}
\setlength{\topmargin}{-0.6in}
\setlength{\textheight}{9.05in}
\setlength{\textwidth}{6.5in}
\newcommand{\plsfill}{{\color{red}XXXFillmeXXX}}
\usepackage{color}
\usepackage{cite}
\usepackage{verbatim}
\usepackage[hyphens]{url}

\usepackage{epsfig}
\usepackage{amssymb}
\usepackage{amsmath}
\usepackage{amsfonts}

\usepackage{slashbox}
\usepackage{graphicx}

%\usepackage[top=0.9in, bottom=0.9in, left=0.75in, right=0.75in]{geometry}
%\usepackage[T1]{fontenc}
%\usepackage{amsmath,  amsthm,  amssymb}
%\usepackage{amsmath,    amssymb}
%\usepackage[pdftex]{hyperref}
%\usepackage[english]{babel}
%\usepackage[dvips]{graphicx}
%\usepackage{algorithmic}
%\usepackage{algorithm}
%\usepackage{listings}
%\usepackage{clrscode}
%\usepackage{styles/algorithmic}
%\usepackage{styles/algorithm}
%\usepackage{styles/listings}
%\usepackage{styles/clrscode}
%\usepackage{clrscode}

\def\backGroundColor{white}
\def\txtsize{\normalsize}
\def\bl{\par\textcolor{\backGroundColor}{\txtsize{ empty line }}\par}
\def\it{\textit}
\def\sur{m}
\def\surit{\it{m}}

\def\includeComments{include}
\def\includ{include}

\def\comm[#1]{\ifx\includeComments\includ  \bl \texttt{\textbf{\textit{Note: #1}}} \par \fi}
\def\inlinecomm[#1]{\ifx\includeComments\includ  \textit{Note: #1} \fi}
\def\dn{$k$ }
\def\sn{$n$ }

\def\ith{$i^{th}$}
\def\jth{$j^{th}$}
\def\tth{$t^{th}$}
\def\d{$d$}
\def\s{$s$}
\def\Sd{\mbox{$S(d)$}}
%\def\S{$S$}
\def\D{$D$}
\def\subS{$S$}
\def\dpr{\mbox{DPR}}
\def\dpsr{\mbox{\em dpsr}}

\def\DM{D}
\def\dM{d}
\def\SdM{S(d)}
\def\subSM{S}
\def\SM{S}
\def\sM{s}
\def\A{A}
\def\xs{X_S}

\def\dx{D_X}
\def\sx{S_X}
\def\sy{S_Y}
\def\Nr{\mathcal{N}}


\def\dif{U}
\def\ws{(n+k)}
\def\Arc{Hypercube+}

\def\R{\mbox{$\widehat{R}$}}
\def\Rc{\mbox{$\widetilde{R}$}}

\def\pk{\frac{k}{n+k}}
\def\pn{\frac{n}{n+k}}

\def\blackbox{\hfill {\vrule height6pt width6pt depth0pt}}
\def\Box{\hfill \framebox(5.25,5.25){}}
\def\qd{\Box}
\def\QD{\blackbox}

\def\sma{SMA}
\def\decsma{DecSMA}

\newcommand{\comments}[1]{\textit{\underline{Note: #1}}}

\newcommand{\MySection}[1]{\section{\textbf{#1}}}
\newcommand{\MySubSection}[1]{\subsection{\textbf{#1}}}

\def\comfat{combined fatness }

\newcommand{\paras}[1]{\smallskip \noindent {\bf #1}}
\newcommand{\para}[1]{\smallskip \noindent {\bf #1}}
\newcommand{\softpara}[1]{\smallskip \noindent \underline{#1}}

\pagenumbering{arabic} \pagestyle{plain}

\newcommand{\F}{\mbox{${\cal F}$}}
\newcommand{\dl}{\mbox{${\delta}$}}
\newcommand{\B}{\mbox{${B_{\dl}}$}}
\newcommand{\e}{\mbox{${\varepsilon}$}}

\newcommand{\red}[1]{\textcolor{red}{#1}}
\newcommand{\blue}[1]{\textcolor{black}{#1}}
\newcommand{\cam}[1]{\textcolor{black}{#1}}
\newcommand{\green}[1]{\textcolor{green}{#1}}


\newcommand{\ommited}[1]{\textcolor{magenta}{#1}}
%\newcommand{\ommited}[1]{}


\newcommand{\cbl}{\color{blue}}
\newcommand{\cbr}{\color{red}}
\newcommand{\cw}{\color{white}}
\newcommand{\cb}{\color{black}}
\newcommand{\uneat}[1]{\textcolor{green}{#1}}
\newcommand{\Eat}[1]{}
\newcommand{\eat}[1]{}
\newcommand{\vsp}{\vspace{0.05in}}

%\newtheorem{algorithm}{Algorithm}
%\newtheorem{definition}{Definition}
%\newtheorem{theorem}{Theorem}
%\newtheorem{lemma}{Lemma}

\newtheorem{proposition}{Proposition}
\newtheorem{formula}{Formula}
\newtheorem{problem}{Problem}
\newtheorem{corollary}{Corollary}
\newtheorem{Plain}{}
\newtheorem{defin}{Definition}
\newtheorem{ex}{EXAMPLE}
\newtheorem{thm}{Theorem}
\newtheorem{lem}{Lemma}
\newtheorem{alg}{Algorithm}
\newtheorem{ob}{Observation}

\newenvironment{theorem}{\vsp \begin{thm} \nopagebreak}{{\hfill$\blackbox$} \end{thm} \vsp}
\newenvironment{thm-prf}{\vsp \begin{thm} \nopagebreak}{\end{thm}}
\newenvironment{lemma}{\vsp \begin{lem} \nopagebreak}{{\hfill$\blackbox$} \end{lem} \vsp}
\newenvironment{lem-prf}{\vsp \begin{lem} \nopagebreak}{\end{lem}}
\newenvironment{prf}{{\sc Proof:} \nopagebreak }{{\hfill$\blackbox$} \vsp}

\newenvironment{definition}[1]{\vsp\begin{defin}\begin{rm}({#1})}
        {{\hfill$\Box$} \end{rm}\end{defin} \vsp}
\newenvironment{algorithm}{\begin{alg}\nopagebreak\begin{rm}
        \begin{tabbing}Tb\=Tb\=Tb\=Tb\=Tb\=Tb\=Tb\=Tb\=Tb\=Tb\=Tb\=Tb\=Tb\=\kill }
        {\end{tabbing} {\hfill$\Box$} \end{rm}\end{alg}}

\Eat{

\newenvironment{example}{\begin{ex} \nopagebreak
  \begin{rm}}{{\hfill$\Box$} \end{rm}\end{ex}}
\newenvironment{observation}{\noindent \begin{ob}}{{\hfill$\Box$}\end{ob}}
}

\newcommand{\squishlist}{
 \begin{list}{$\bullet$}
  { \setlength{\itemsep}{0pt}
     \setlength{\parsep}{3pt}
     \setlength{\topsep}{3pt}
     \setlength{\partopsep}{0pt}
     \setlength{\leftmargin}{1.5em}
     \setlength{\labelwidth}{1em}
     \setlength{\labelsep}{0.5em} } }

\newcommand{\squishlisttwo}{
 \begin{list}{$\bullet$}
  { \setlength{\itemsep}{0pt}
     \setlength{\parsep}{0pt}
    \setlength{\topsep}{0pt}
    \setlength{\partopsep}{0pt}
    \setlength{\leftmargin}{2em}
    \setlength{\labelwidth}{1.5em}
    \setlength{\labelsep}{0.5em} } }

\newcommand{\squishend}{
  \end{list}  }



\newcommand{\BEGIN}{{\bf BEGIN\ }}
\newcommand{\END}{{\bf END.}}
\newcommand{\Begin}{{\bf Begin\ }}
\newcommand{\End}{{\bf End}}


\newcommand{\Do}{{\bf do}}
\newcommand{\Else}{{\bf else}}

\newcommand{\Endif}{{\bf end if;}}
\newcommand{\Endfor}{{\bf end for;}}
\newcommand{\Endwhile}{{\bf end while;}}

\newcommand{\For}{{\bf for}}
\newcommand{\for}{{\bf for}}
\newcommand{\If}{{\bf if}}
\newcommand{\In}{{\bf in}}
\newcommand{\Let}{{\bf let}}
\newcommand{\Repeat}{{\bf REPEAT\ }}
\newcommand{\Return}{{\bf RETURN\ }}
\newcommand{\Then}{{\bf then}}
\newcommand{\Until}{{\bf UNTIL\ }}
\newcommand{\While}{{\bf while}}
\newcommand{\When}{{\bf When}}
\newcommand{\On}{{\bf On}}
\newcommand{\Procedure}{{\bf Procedure}}
\newcommand{\Break}{{\bf break}}


\newcommand{\Max}{{\bf max}}
\newcommand{\Min}{{\bf min}}

\newcounter{packednmbr}
\newenvironment{packedenumerate}{\begin{list}{\thepackednmbr.}{\usecounter{packednmbr}\setlength{\itemsep}{0pt}\addtolength{\labelwidth}{-5pt}\setlength{\leftmargin}{\labelwidth}\setlength{\listparindent}{\parindent}\setlength{\parsep}{0pt}\setlength{\topsep}{3pt}}}{\end{list}}
\newenvironment{packeditemize}{\begin{list}{$\bullet$}{\setlength{\itemsep}{0pt}\addtolength{\labelwidth}{-5pt}\setlength{\leftmargin}{\labelwidth}\setlength{\listparindent}{\parindent}\setlength{\parsep}{0pt}\setlength{\topsep}{3pt}}}{\end{list}}


%\newcommand{\Section}{Section~}
\newcommand{\Section}{\S}

\usepackage{xspace}

\newcommand{\DC}{DC\xspace}
\newcommand{\DCs}{DCs\xspace}

\usepackage[font=bf,aboveskip=7pt,belowskip=-10pt]{caption} % SPACE
\newcommand{\tightcaption}[1]{\caption{#1}}
%\newcommand{\tightcaption}[1]{\caption{\em #1}}

\newcommand{\vyas}[1]{{\footnotesize\color{magenta}[VS: #1]}}

\begin{document}

\centerline{\LARGE\bf Collaboration Plan}
\bigskip

\section{Team Expertise}
The project will be collaborative between Stony Brook University (SBU)
and Carnegie Mellon University (CMU). 
The project team has a broad range of  inter-disciplinary expertise available 
suitable to deliver on different facets of the project.
\begin{itemize}

\item PI Gupta (Assoc. Professor, Computer Science, Stony Brook University)  is
an expert in algorithmic aspects of networking, specifically involving routing,
scheduling, connectivity and geometric aspects mostly related to wireless and
sensor networks. 

\item PI Sekar (Asst. Professor, ECE, Carnegie-Mellon University) has interest
in broad aspects of networked systems. His work spans design and management of
middleboxes, Internet video, scalable network monitoring, content distribution,
software-defined networking and network forensics.  

\item Co-PI Das (Professor, Computer Science, Stony Brook University) is an
expert in networking protocols and systems – specifically focusing on
wireless/mobile networks. He has built several experimental systems related to
ad hoc/mesh and sensor networking and vehicular networking. In prior research,
he also developed scalable discrete event simulation  systems.  

\item Co-PI Longtin (Professor, Mechanical Engineering, Stony Brook University)
is an expert in laser ...  \plsfill.

\end{itemize}

A total of 4 PhD students (2 CS students and 1 ME student in SBU and 1 ECE student
in CMU) will participate in the project. Each will be primarily advised by a
PI/Co-PI.  They will also closely work with the rest of the team. 

\begin{wraptable}{r}{0.5\textwidth}
\begin{small}
\begin{tabular}{p{2cm}|p{2cm}|p{3cm}}
Task	& 	 Lead	&  Collaborator(s) \\ \hline
FSO1	& 	Das		& Longtin \\  
FSO2	& 	Longtin		& Das \\ 	
Theory1	&	Gupta		& Sekar \\
Theory2	& 	Gupta		& Sekar \\
System1	&	Sekar		& Gupta \\
System2	&	Sekar		& Das \\ 
System3	&	Das		& Sekar \\
E2E	&	Das, Gupta	& Longtin, Sekar 
\end{tabular}
\end{small}
\caption{Task leads}
\label{table:tasklead}
\end{wraptable}

While all investigators will participate in all parts of the project, each
Task will be lead by one of the investigators (see Table~\ref{table:tasklead}). 
Gupta will lead Tasks \plsfill that are mostly related to topology design
and algorithms for traffic engineering. Sekar will lead Tasks \plsfill that are 
related to network management and control plane design. Das will lead
Tasks \plsfill that related to system building and evaluations. Longtin will lead 
Tasks \plsfill related to building the lower layers of FSO-based design. 

\section{Division of Work and Collaboration Plan}

The PI/Co-PIs have history of collaborations. Gupta and Das have long collaborated 
on several wireless/sensor networking topics and have published 15+ papers~\cite{} together. They  also
collaborated in 4 different NSF-funded projects.  The collaborations with Sekar started with Sekar 
joining SBU faculty in 2012. This collaboration resulted in the preliminary work along the 
direction of this proposal -- published as a position paper in ACM HotNets 2013~\cite{}. 
After this initial success, the team sought input from Longtin who helped build 
the SFP-based FSO link (Figure \plsfill in the proposal) in his optics lab thus demonstrating
the general feasibility of the design. 
Das and Longtin also collaborated in a home energy monitoring project~\cite{} funded by 
NYSERDA.

With Sekar joining CMU faculty starting 2014, the collaborations will continue  
remotely aided by frequent `skype' meetings and semi-annual visits. 
%
The hardware prototype will be primarily developed in SBU and has been
accordingly budgeted. The initial development and characterization of the  FSO
links, the alignment and steering techniques will be pursued in the optics lab
(in SBU Mechanical Engineering) that Longtin directs. The lab has excellent set
up for laser studies including multiple optical benches, \red{say something
about the lab's equipments that can benefit the project}. Concurrently, the
simulation platforms will be set up and trace-driven simulations will be
performed  on the initial topology control and reconfiguration techniques (both
in SBU and CMU by Gupta and Sekar in their respective labs). At this time, the
NetFPGA-based switches will be developed as well (in SBU WINGS/PCL Lab by Das)
to aid the final end-to-end prototyping.  The SDN-based control plane will be
primarily developed in CMU (by Sekar) and will be ported to SBU in time for the
end-to-end testing. The RF-based controller system will developed in SBU (in
SBU WINGS/PCL Labs by Das).  The general description of above labs and other
facilities available for the project are described in the Facilities section. 

In the third year of the project, we expect the initial design of FSO and
steering mechanisms will have stabilized and understanding of the rest of the
Firefly system will have progressed far enough. At this time the  project will
concentrate significantly on the end-to-end prototyping. This will be done in
the CEWIT datacenter (as articulated in Section \plsfill) in SBU (led by Das).
The large-scale evaluations and refinement of topology control, reconfiguration
and control plane will still continue while this proof-of-concept prototype is
developed.  \vyas{it might be better for the lead PI to be on the integration task?} 


\section{Project Timeline}

\red{[sd: Below needs revision. Best to state in terms of tasks/subtasks.]}

\newcommand{\myc}{$\circ$}

\begin{table}[h]
\begin{center}

\begin{tabular}{l||c|c||c|c||c|c||c|c}
		& \multicolumn{2}{c||}{Year 1 (2014)} & \multicolumn{2}{c||}{Year 2 (2015)} & \multicolumn{2}{c||}{Year 3 (2016)}  & \multicolumn{2}{c}{Year 4 (2017)}\\
					& Fall & Spring & Fall & Spring & Fall & Spring & Fall & Spring \\ \hline
 \taskref{task:system:fastalgo}: Algorithms 	&	&  & & \myc &  \myc  &       \myc   & \myc &  \\
 \taskref{task:system:dataplane}: DataPlane 	&	&  & & \myc &  \myc  &       \myc   & \myc &  \\
 \taskref{task:system:ctrlchannel}: ControlChannel 	&	&  & & \myc &  \myc  &       \myc   & \myc &  \\
 \taskref{task:eval:demo}: E2E Demos 	&	&  & & \myc &  \myc  &       \myc   & \myc &  \\
\end{tabular}

\end{center}

\caption{Projected schedule for tasks described in the previous sections. The $\circ$ shows when 
 a task is ``active''.  Some tasks are  split in the timeline as we 
 will need to revisit/integrate with respect to other aspects of the proposed work. 
 }

\label{tbl:schedule}
\end{table}


\end{document}
