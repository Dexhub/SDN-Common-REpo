\section{Possible Extensions to \ArchName Architecture}
\label{sec:ext}

In our project, we would also investigate the following ideas, which
impart more flexibility to our design and/or present additional
opportunities to improve \ArchName's performance.

\begin{packeditemize}
\item
{\em Dynamic Reconfiguration of Link Bandwidths.}  In \ArchName, the
\blue{bandwidth} of each FSO link is limited by the capacity of the
ToR switch port. Facilitating FSO links to have variable bandwidth
would add another dimension of flexibility to our design. 
%
This can be achieved by having each port of ToR switch associated with
a unique wavelength, and using a multiplexer and a WSS (wavelength
selective switch) unit between the ToR switch and the FSOs, as in
OSA~\cite{osa-tr}. In essence, the multiplexor and WSS allow one or
more ToR ports to ``feed'' into a single FSO link, and hence,
facilitating variable-bandwidth FSO links. The WSS can be reconfigured
in real-time to determine the allocated ports (and hence, bandwidth)
to each FSO link.
%
It would be interesting and challenging to incorporate the above
flexibility into our design, and appropriately generalize the
techniques from Section~\ref{sec:topology} and~\ref{sec:system}.

\item
{\em Non-ToR Switches and FSOs.} As described before, our network
architecture consists of only ToR switches whose ports are either
connected to the rack machines or the FSOs on the rack. Incorporating
non-ToR switches (as in most data center architectures~\cite{}), whose
ports are connected to only FSOs, can impart more flexibility to
\ArchName's design. The new challenges that need to be addressed in
this context are: (a) Physically accomodating the FSOs connected to
such non-ToR switches in the data center, (b) PCFT, \BBO, and JRTE
problems would include determining the inter-connections between these
additional switches. Note that the resulting PCFT solutions would now
be non-regular graphs.

\item
{\em Vertically Steerable FSOs; $45^\circ$ Mirror Poles.} In our
design, we use a ceiling mirror to circumvent physical obstruction for
line-of-sight FSO communication. However, in certain contexts such as
outdoor scenarios for containerized architectures~\cite{dc-container}
installing a ceiling mirror may not be feasible. In such cases, we
need other mechanisms for line-of-sight communications.  E.g., we can
install FSOs on vertically-steerable poles such that each link
operates on a separate horizontal plane. Another possible mechanism
could be to have FSOs direct their beams to vertically-steerable small
mirrors angled at $45^\circ$. To avoid physical obstruction from
poles/mirrors in the above approaches, shorter distance links can be
operated on a lower horizontal plane than the longer distance links.
\end{packeditemize}

\eat{
\item
{\em Multicast.} Big data applications have diverse communication
patterns that mix together unicast, multicast, all-to-all cast,
etc. Recent works show~\cite{ccr-8} show that all-to-all data exchange
on average accounts for 33\% of the runnign time of Hadoop jobs. As
suggested in~\cite{ccr}, the optical communications are particularly
amenable to efficient implementation of such *-cast patterns by
leveraging various components such as directional couplers,
wavelength-division multiplexed, etc. \blue{Incorporating the above
  ideas in our design will require making challenging design choices.}
}
