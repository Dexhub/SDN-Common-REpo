\section{\ArchName Network Management}
\label{sec:system}

In this section, we focus on the design of a {\em datacenter
  management layer} that uses building blocks from previous sections,
to implement a \blue{feasible} reconfigurable datacenter network.
%
We start with describing the high-level roles of the different
components of the management layer. See Figure~\ref{fig:mgmt}.

\begin{wrapfigure}{r}{0.5\textwidth}
\vspace{-1cm} 
\includegraphics[width=240pt]{PPTFigs/sysoverview-new.pdf}
\vspace{-0.8cm} 
\caption{Overview of the \ArchName management layer}
\vspace{-0.6cm} 
\label{fig:mgmt}
\end{wrapfigure}

\begin{packeditemize}
\item {\bf Monitoring Engine (ME):} ME provides network status
  information (e.g., links status, traffic patterns, or views
of ``elephant'' flows~\cite{hedera,devoflow,conext13ericson})
to the management layer. 

\item {\bf Optimization Engine (OE):} Given the offered traffic
  workload, a pre-configured flexible topology (PCFT), and the current
  network state (e.g., link status), the optimization engine devises
  an efficient {\em reconfiguration and traffic engineering strategy}
  so as to optimize desired performance goals.

\item {\bf Data Plane Translation Engine (DPE):} DPE translates the
  OE output into an efficient data plane strategy.

\item {\bf Application APIs:} APIs enable the users/tenants to inform
    application details (e.g., expected traffic patterns,
    single/multi-path TCP) to the optimization and data plane modules,
    \blue{to best leverage the benefits of \ArchName.}
\end{packeditemize}

\softpara{Plan.} For the ME, we will leverage past work on scalable 
traffic matrix and elephant flow
detection~\cite{hedera,devoflow,yingzhangconext13}.  Similarly, we
will extend prior work on APIs for applications to expose traffic
patterns~\cite{coflow,bigdatahotsdn}.
%
We address OE and DTE in the following subsections. In the last
subsection, we address design of a wireless out-of-band control
network for configuration dissemination and data collection.

\subsection{Reconfiguration and Traffic Engineering}
\label{sec:reconfig}

\begin{task}
\label{task:system:fastalgo}
We will develop fast and efficient algorithms for the joint
optimization problem of reconfiguration and traffic engineering.
\end{task}

%To explain the optimization problem that we
%need to solve and highlight why this problem is hard, we begin with an abstract
%integer linear program formulation in a static setting.

\para{Joint Reconfiguration and Traffic Engineering (JRTE) Problem.}
Given the traffic load, the {\em JRTE} problem is to: (a) {\em
  Reconfigure}, i.e., select a realizable-topology from a given
pre-configured flexible topology, and (b) {\em Traffic Engineer (TE)},
i.e,.  route the given inter-rack flows over the realizable-topology
selected in (a), to optimize a desired objective.
%
The objective functions of interests could be to minimize link
congestion or maximize total flow in conjunction with fairness,
latency bound, and/or tenant provided SLAs.

The reconfiguration subproblem of JRTE falls in the general class of
{\em degree-constrainted subgraph} problems~\cite{}, and is akin to
the topology control problem~\cite{} addressed for wireless
networks. However, the constraints and TE-based objectives of our
subproblem makes it very different prior-addressed problems. Note that
the TE subproblem of JRTE is essentially solving the NP-hard
unsplittable multi-commodity flow problem~\cite{hedera,mcf} over the
selected topology---thus, the JRTE problem is trivially NP-hard.

\para{Proposed Approaches.} Note that the JRTE algorithm is run in
real-time, and hence, should not take more than a few milliseconds,
for it to be of any real benefit. Thus, the challenge is to design
very fast, scalable, and efficient JRTE algorithms.

\begin{packeditemize}
\item {\em Matching Techniques.} Recall from Section~\ref{sec:pcft}
  that a realizable-topology is a matching over the FSOs. Thus, a
  reasonable approach to the reconfiguration subproblem could be to
  select the maximum-weighted matching (solvable in polynomial time)
  between FSOs, where the link $(i,j)$ is weighted by the inter-rack
  traffic demand between the correspondin racks. Such a topology
  maximizes the total inter-rack demand that can be served using one
  hop. In general, we would like to pick a matching that yields the
  minimum weighted-average inter-rack distance (where the distances
  are weighted by the traffic demands); however, standard matching
  techniques do not apply to this unsubmodular~\cite{} objective
  function. One approach would be to use a greedy strategy that
  iteratively adds augmenting paths~\cite{} based on a suitable
  benefit-function for augmenting paths. After reconfiguration as
  above, the TE part can be done using known techniques~\cite{} over
  the selected matching.

\item {\em LP Relaxation Techniques.} Another promising approach is to
  formulate the JRTE problem as an ILP (using splittable flow-like
  constraints, and binary variables for link selection) with the
  objective of minimizing link congestion or maximum ``fair''
  flow~\cite{max-fair-flow}, and solve the relaxed LP. We can then
  convert the LP solution to an ILP solution by an appropriate
  rounding technique while ensuring that the matching-constraint is
  still satisfied (unsatisfied flow-constraints only result in a
  sub-optimal but valid solution). \blue{To handle the unsplittable
    aspect of flows, we can: (i) use the classic path-striping
    technique~\cite{rand-round}, and/or (ii) allow flows to be split,
    but minimize the number of splits by considering the {\em
      aggregate} flow per destination, as suggested in the recent
    work~\cite{localflow}.} An alternate approach in a similar vein as
  above is: First solve the TE problem over the entire PCFT graph, and
  use the flow-values to select a ``good'' matching. It would be
  interesting to compare the performance of above approaches over real
  traffic traces.
\end{packeditemize}

\para{Further Directions.} In addition to the above, we are also
interested in designing algorithms based on limited traffic
information, and incremental or localized approaches.

\begin{packeditemize}
\item {\em Strategies with Limited Traffic Information.} Our previous
  discussion implicity assumes availability of traffic demands for the
  next \blue{epoch.}  However, in reality, traffic predictability may
  be limited. In the worst case, we may only be able to distinguish
  between ``elephant'' (large) and ``mice'' (small) flows, based on
  their \blue{initial size.}  In such restricted settings, a
  reasonable approach would be to solve the JRTE problem only in
  response to the arriving elephant flows, while relying solely on TE
  for the mice flows. In our preliminary work~\cite{hotnets}, we
  employed a simple strategy based on the above idea, and achieved
  near-optimal performance over randomly generated traffic
  traces. More information about traffic patterns such as spatial and
  temporal distribution of elephant flows (or flow sizes in general)
  would require challenging generalizations of the above
  approach. Such elephant flow based approaches should be quite
  effective, since the structure of real workloads suggests that
  elephant flows are typically long-lived~\cite{} and a small number
  of them carry the most bytes~\cite{}. An additional objective, in
  the context of above approach, should be to favor JRTE solutions
  that cause minimal disruption to existing traffic flows. This could
  be achieved by developing techniques to compute the ``impact'' of a
  solution on the existing traffic; this may be intractable, but
  computing upper and/or lower bounds of the impact may be sufficient
  for our purposes.

\item {\em Incremental or Localized Strategies.} One way to develop
  fast JRTE algorithms is to determine the solution in an incremental
  manner (e.g., by constraining the number of links that are activated
  or deactivated). Morever, we can also restrict ourselves to only
  ``localized'' distributed strategies. To design incremental JRTE
  algorithms, we can use augmenting-path techniques to incrementally
  improve the matching~\cite{}, with additional constraints and/or a
  benefit function (as suggested before). \blue{To design localized
    strategies, we would explore extend the recent
    work~\cite{localflow} which develops a simple switch-local routing
    algorithm. In our context, we will locally (e.g., at each switch)
    change active links as well as the forwarding hops, in response to
    sufficient change in arriving flows.}
\end{packeditemize}

\subsection{Data Plane Strategies}

\begin{task}
\label{task:system:dataplane}
We will design and implement efficient data-plane implementations to
guarantee \blue{reachability and consistency properties}, in presence
of reconfiguration and link dynamics.
\end{task}

In translating the solution provided by the optimization engine into a
consistent and efficient data plane forwarding strategy, We build upon
recent advances in software-defined networking (SDN).  While SDN is an
``enabler,'' as it provides cleaner management abstractions and open
interfaces (e.g., via APIs such as OpenFlow~\cite{}), \ArchName
introduces unique consistency and efficiency challenges.  In
particular, in face of a dynamically changing network due to
reconfigurations, we need to ensure that (a) packets do not use
deactivated links \blue{(i.e., black holes~\cite{} are avoided)}, (b)
the network remains connected at all times, and (c) the packet latency
remains bounded. Finally, we also need a data plane strategy to: (d)
handle transient misalignment of FSO links. We note that the recent
related works either assume a {\em static}
network~\cite{cons-update,incconsupdate} or focus on a single
reconfiguration~\cite{cu-1}, and hence, are not directly applicable to
our context.

\para{\underline{(a) and (b).} Guaranteeing Correctness and Connectivity.}
Packets are routed in the network on the basis of forwarding tables,
which essentially specify, at each node, the next hop/link to use for
each destination. In a dynamic network, forwarding tables will also be
changing constantly.  Note that activation/deactivation of a link
takes a finite amount of time, and that we cannot update the tables
across all the network switches atomically (i.e., {\em at once}).
%
In face of the above challenges, we need to ensure that through every
possible intermediate state of the links and switches' tables, only
active links appear in the forwarding tables. We can ensure this by a
careful ordering of steps as suggested in our preliminary
work~\cite{hotnets}; in particular, (i) we reflect removal of links in
the forwarding tables, before actually deactivating the links, and
(ii) reflect addition of links in the tables only after the link
activation is complete. Note that the above solution ensures the
desired property even in face of multiple concurrent reconfigurations,
and irrespective of the order in which forwarding tables are updated
across the network.

In addition to above, we also need to ensure that the network remains
connected at all times. There are two possible options: (i) We
maintain a static ``backbone'' subnetwork that ensures connectivity,
or (ii) reject reconfigurations that disconnect the network. The first
approach reduces the degree of flexibility in network design and
\blue{may result in high packet latency, depending on the backbone.}
The second approach becomes challenging to implement if there are
multiple {\em concurrent} reconfigurations. There are three options to
handle concurrent reconfigurations: (i) one at a time, (ii) in batches
(i.e., queue and combine them into a single reconfiguration); and
(iii) execute each reconfiguration individually but {\em
  concurrently}. The first two options can be inefficient as large
flows have to wait until the desired link(s) become available, while
the third option requires a careful implementation to ensure
consistency. In particular, for the third option, we need to keep a
single consistent view of the network topology graph and allow only
{\em atomic} access to it (when one needs to check if deactivation of
a set of links disconnents the network). \green{Again, the above
  solutions work irrespecitve of the order in which the forwarding
  tables are updated across the network.} In our research, we will
study the performance of the above described approaches.

\begin{wrapfigure}{r}{0.3\textwidth}
\vspace{-0.4cm}
\centering
\includegraphics[width=150pt]{PPTFigs/impossible.pdf}
\caption{If the network goes back and forth between the above
  topologies (due to corresponding reconfigurations), then the packet
  will continue to ``swing'' between areas A and B -- leading to a
  forwarding ``loop''.}
\label{fig:impossible}
\end{wrapfigure}
\para{\underline{(c)} Guaranteeing Bounded Packet Latency.}  The above
strategies still do not guarantee a bounded packet latency. In fact,
in general, its {\em impossible} to avoid guarantee bounded packet
latency in general. See Figure~\ref{fig:impossible}. There are two approaches to
bound the latency of (in-flight) packets: (i) create and use a
backbone {\em static} subnetwork with bounded diameter, (ii) reject
reconfigurations to avoid high packet latency. The first approach will
require careful decision-making of {\em when} to resort to routing a
packet to the backbone (due ot its limited bisection bandwidth), while
the second approach will require an efficient and fast computation of
the impact on latency of the packets (especially, the in-flight
packets). In addition, we should formally characterize and avoid 
scenarios like the one described in Figure~\ref{fig:impossible}.

\para{\underline{(d)} Handling Misalignment of Links.} In \ArchName, even during
a static topology state, links may be temporarily unavailable because
of possible misalignment of the FSO links. \blue{Such misalignments
  are fixed in real-time by ``micro alignment'' of FSO devices, as
  suggested in Section~\ref{sec:fso}}, and the timescales of such
  micro-alignment is likely to be much smaller than the time needed to
  update rules~\cite{ddcnsdi13} through an SDN controller. In fact, it
  may even be counterproductive to report such transient link failures
  to the controller, as it may cause needless \blue{reconfigurations
    and/or update of forwarding tables}. Thus, we need appropriate
  \blue{network layer} techniques to recover from such transient link
  failures. \blue{Future SDN roadmaps have provisions for local
    recovery mechanisms analogous to similar schemes in the MPLS and
    SONET literature~\cite{}. We will explore the available
    alternatives in our research. In the absence of such features,} we
  will investigate design of a local ``lightweight'' SDN controller on
  every rack that can quickly react to such misalignments while
  relying on the global controller for longer-timescale
  reconfigurations~\cite{ddc}.

