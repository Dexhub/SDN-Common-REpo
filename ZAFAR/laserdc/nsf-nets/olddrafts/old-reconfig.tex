
\subsection{Fast reconfiguration and traffic engineering}

\begin{task}
\label{task:system:fastalgo}
We will develop fast, near-optimal algorithms for  topology
reconfiguration and traffic engineering. We will  investigate tradeoffs between
  performance goals  using real-world datacenter traces and
application requirements. We will also investigate distributed and online
algorithms  that are robust to estimation errors and do not
require global coordination.  
\end{task}


%\begin{task}
%\label{task:system:onlinelocal}
%\end{task}


%
\newcommand{\NumRacks}{\ensuremath{\mathit{N}}} 
\newcommand{\NumFSOs}{\ensuremath{\mathit{M}}} 
\newcommand{\Rack}{\ensuremath{\mathit{Rack}\xspace}} 
\newcommand{\FSO}{\ensuremath{\mathit{FSO}\xspace}} 
\newcommand{\RackIndex}{\ensuremath{\mathit{i}}} 
\newcommand{\FSOIndex}{\ensuremath{\mathit{j}}} 

\newcommand{\RackFSOa}{\ensuremath{\RackIndex,\FSOIndex}} 
\newcommand{\RackFSOb}{\ensuremath{\RackIndex',\FSOIndex'}} 

\newcommand{\RackFSOc}{\ensuremath{\RackIndex,\FSOIndex}} 
\newcommand{\RackFSOd}{\ensuremath{\RackIndex,\FSOIndex}} 



\newcommand{\Flow}{\ensuremath{\mathit{f}}} 
\newcommand{\InterRack}{\ensuremath{\mathit{k}}}
\newcommand{\Traffic}{\ensuremath{\mathit{T}}}
\newcommand{\Edge}{\ensuremath{ (\RackFSOa \rightarrow \RackFSOb) }} 
\newcommand{\EdgeRev}{\ensuremath{ (\RackFSOb \rightarrow \RackFSOa) }} 
\newcommand{\Edgefoo}{\ensuremath{ (\RackFSOc \rightarrow \RackFSOd) }} 
\newcommand{\EdgeVar}{\ensuremath{\mathit{d}}}
\newcommand{\EdgeVarIndex}{{\EdgeVar_{\Edge}}}
\newcommand{\EdgeVarRevIndex}{{\EdgeVar_{\EdgeRev}}}
\newcommand{\Capacity}{\mathit{C}}
\newcommand{\FlowEdge}{\ensuremath{\Flow_{\InterRack,\Edge}}} 



 
%To explain the optimization problem that we
%need to solve and highlight why this problem is hard, we begin with an abstract
%integer linear program formulation in a static setting.
\mypara{Problem Context and Challenges} At a high-level, we need to solve a a
joint topology design and TE problem in contrast to prior
work in network management that considers TE  in isolation on
fixed topologies.  We consider a setting with $\NumRacks$ {\Rack}s each
equipped with $\NumFSOs$ {\FSO} devices.   Let the index $\RackIndex$  refer to
a specific {\Rack} and  $\FSOIndex$  refer to a specific {\FSO} device.  Thus,
the subscript $\RackFSOa$  will refer to a specific Rack-FSO combination.  Let
$\InterRack$ denote a pair of racks and let us assume that we know some
estimated traffic matrix for every pair of racks denoted by
$\Traffic_\InterRack$ for the volume of traffic from $\InterRack$.  

%Suppose the bandwidth of each inter-FSO link (if enabled) is $\Capacity$. 

 As a starting point, we consider the objective  on  routing   the demands in
the traffic matrix to  minimize network  congestion~\cite{thorupfortz,jrexte}.
 There are  two main level of control decisions. First, we need to decide
 if a pair of Rack-FSO links need to be activated; $\EdgeVarIndex$ is 
a binary variable that is 1 if we choose to activate  $\Edge$. 
 Note that  we can only choose from a set of
$\Candidates_{\RackFSOa}$ possible links as defined by the pre-configured
topology from Section~\ref{sec:topology}.  Second, we need a routing 
 strategy for each demand; let $\FlowEdge$
 denote the volume of the inter-rack  $\InterRack$ demand routed on the
edge $\Edge$.  Given these control variables, we can model the problem as a
Integer Linear Program (not shown) that combines {\em edge activation}  
 with traditional max-flow like constraints for {\em flow routing}. 


%\mypara{Challenges}

 Unfortunately, this problem is theoretically NP-hard (not shown due to space
constraints) and it is  practically intractable---state-of-art ILP solvers take
several hours even for a 20-node problem instance.   In practice, we also  need
to model metrics  such as fairness, latency, as well as tenant-provider SLAs.
Furthermore, this implicitly ssumes that (a) the traffic demands are known (or
predictable) and (b) that the optimization can be run by a global optimizer.


%$two  natural
%c%lasses of  approaches: (1)  ``greedy'' solutions especially if the specific
%bjective functions are submodular~\cite{} and (2)

\mypara{Proposed Approach} We need to a scalable optimization across  multiple
metrics that is also amenable to a distributed/online implementation.  We
propose to explore three complementary strategies: 

\begin{packeditemize}

\item {\em Approximation algorithms:} Specifically, we will investigate
``randomized rounding'' strategies that solve a relaxed linear program (i.e.,
converting the discrete $\EdgeVarIndex$ into fractionals)  and then selectively
choosing to ``round'' some of these fractions to 1. We can also exploit  the
``flow'' structure of the optimization to design distributed optimization
strategies~\cite{khandekar,conext13}.

% However, these existing algorithms cannot be
%directly applied as they do not address the topology selection aspects.

% For instance, we can scope the problem to 
%  only look at the traffic matrix entries with the highest demands 
% or alternatively constrain the number of links that will be 
% reconfigured.  Second, we will explore   Similarly, 

\item {\em Multi-stage optimization:} Intuitively, solving the traffic
engineering problem is simpler   than the joint optimization; e.g., using
max-flow solvers~\cite{}. Thus, one  heuristic is to see if the new traffic
demand can be satisfied by better TE over the existing
topology configuration and  check if this TE solution is  close to the optimal.   At first glance, this is  a
chicken-or-egg problem because we need to run the hard optimization to
determine how close to optimal we are!  In practice, we only need a bound on
the optimal value and not the solution and we can run a relaxed LP to get an
upperbound.

%\footnote{In case of maximization problems,
%the LP will be a upperbound and in case of minimization problems the LP
%solution will be a lower bound.} and compare the solution provided by the
%traffic engineering optimization to this bound.

\item {\em Exploit real-world structure:} We can leverage the natural structure
of real-world workloads; e.g.,  a small number of ``elephant'' flows carry the
most bytes~\cite{}.  Since these elephant flows are typically
long-lived~\cite{}, they are amenable to coarser time-scale optimizations.
Thus, we can design heuristics that use topology reconfiguration logic for the
elephants and fallback to TE for the ``mice''.  This structure is also
naturally amenable to an online/local strategy as it avoids the need for global
coordination.


\end{packeditemize}


