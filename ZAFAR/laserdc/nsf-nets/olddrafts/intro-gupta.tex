\section{Introduction}

%% datacenter is hot
Data centers (DCs) are a critical piece of today's computing infrastructure
that drive  key networked applications.
In this context, data center network designs must satisfy several potentially
conflicting goals---performance (e.g., minimize oversubscribed links,
low latency)~\cite{fattree,vl2}, equipment and management
cost~\cite{fattree,popa-cost}, flexibility to adapt to changing
traffic patterns~\cite{proteus,3db,flyways,cthru}, incremental
expandability to add new servers or racks~\cite{legup,jellyfish}, and
other practical concerns including cabling complexity, power, and
cooling~\cite{farrington,portland,cabling}.
%
The traditional data center architectures are fixed (wired) and hence,
are over-provisioned to handle the worst-case communication
requirements rather than the average-case, increasing the cost and
maintenance of networking architecture.
%
Some recent works have suggested to augment the simple wired
infrastructure with additional inter-rack wireless~\cite{flyways,3db}
or optical links~\cite{cthru}, to impart certain flexibility to the
network and thus, help alleviate congested hotspots.
%
However, the overall vision of a fully wireless and a largely
flexibible network architecture for data centers that offers favorable
cost-performance tradeoff remains far from realized. The goal of our
research proposal is to realize this vision. 


\para{Our Proposal.}  In particular, we take the flexibility offered
by wireless inter-rack links to the logical extreme and propose to
design an \emph{all-wireless inter-rack data center fabric}. Our
vision, if realized, would provide unprecedented degrees of
flexibility for data centers. That is, it will allow operators to
dynamically reconfigure the {\em entire} network topology to adapt to
changing traffic demands. Moreover, a wireless architecture eliminates
maintenance and operational costs due to cabling complexities, and
more importantly, can act as an enabler for operators to explore and
benefit from topology structures that would otherwise remain ``paper
designs'' due to the perceived cabling complexity.

To realize our vision of an \emph{all-wireless} inter-rack data center
fabric, we look beyond the traditional RF-based solutions and explore
a somewhat non-standard ``wireless'' technology, namely {\em
  Free-Space Optics} (FSO).  FSO uses visible or infra-red lasers to
implement point-to-point data links, at very high data rates
($>$1~Gbps) and at longer ranges. Unlike the RF wireless technologies,
FSO communication links have a minimal wireless-interference footprint
(due to their low wavelength) and their usage is not constrained by
federal regulations.
%
While FSO is an enabling technology, there are various scientific
challenges that need to be addressed before we can realize a
cost-effective and flexible DC network fabric. Our research proposal
plans to address these challenges, and build an FSO-based DC network
prototype.

\para{Proposal Objectives, and Our Team.}  In the context of realizing
the above vision of an FSO-based all-wireless inter-rack fabric for a
data center, our research proposal has the following specific
objectives.

\item
We will design an FSO communication device that is cost-effective and
has a small-form factor, for use in a data center network. The device
will include an appropriate steering and alignment mechanism with
effective speed and precision; we will explore feasibility of two
possible steering mechanisms in our research.

\item
We will design efficient algorithms for two topology design problems
that arise in our context: (a) Pre-configuration topology design,
which corresponds to an initial alignment/configuration of the FSO
devices, based on various constraints and available traffic
statistics, and (b) Reconfiguration algorithm, which controls the
real-time configuration of FSO devices based on the current network
traffic and flows.

\item
We will address various network management challenges that arise in
our context. In particular, $\ldots$

\item
We will build a prototype of an FSO device, develop a system prototype
over XX platform, and conduct extensive simulations over real data
center traffic to demonstrate viability and performance of our overall
vision. 
\end{itemize}

Our research consists of three computer scientists with complementary
networking and theoretical expertise and a mechanical engineer with
expertise in laser-based optical measurement techniques, giving us a
complete spectrum of expertise needed for various scientific aspects
of our research proposal.

\para{Broader Impact.}
\vspace*{1in}


