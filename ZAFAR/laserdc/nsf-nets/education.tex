\section{Broader Impact}
\label{sec:edu}

\red{Some input from Jon would be helpful too.}

\para{Impact on Economy and Environment.} 
With growing interest in
Big Data, cloud computing and virtualization, data centers are now
common in every sector of the economy. This includes IT industry,
government, media, healthcare, financial sector, transportation
and the scientific community. 
The largest of the data centers are known to cost more than a billion USD
and are significant power hogs consuming 10s of MW of power~\cite{}. 
Overall, recent EPA studies concluded the the total data center electrical power usage is 
roughly a few percent of the entire electricity consumption in the US
and lagging only modestly behind the total household electricity consumption~\cite{}. 
We foresee that the Firefly architecture can significantly reduce both cost 
(by eliminating the need for over-provisioning)  
and energy consumption (by making the network design
energy-proportional and also by improving cooling). 
This certainly will have perceptible economic impact by making many
IT services cost less - both in terms of dollars and carbon footprint - across all sectors in the economy. 
In addition, success in the proposed project will garner immediate interest in industry 
for further developing and productizing the proposed FSO-based interconnection.
R\&D and manufacturing of such interconnections will produce 
a different form of device industry that will include optical engineers
in addition to traditional computer hardware engineers. 


%
%Performance of data centers, energy savings
%(energy proportionals DCs), cabling complexity, broaden the current
%applications of FSO communication (which is currently limited to specialize
%across-town applications), ad hoc deployment of interconnection architectures 
%(e.g, for Helios type containerized DCs), 

\paragraph{Integration of Research and Education.}

 A strength of the project is that it brings together two disparate disciplines,
opto-electronics and computer systems. 
Due to this unique nature, the participating ME students will learn
basics of data center networking and the CS students will be exposed
to laser communications. From
preliminary studies leading to this proposal -- that involved several grad students from both CS and ME -- 
it is our experience that 
the students particularly enjoyed learning a completely
new technology and understanding the methods and practices of a
different discipline. 

The project will 
directly contribute to relevant graduate courses in both mechanical engineering (ME) and
computer science (CS) that the PIs teach regularly. These include
advanced courses on ``wireless networking'' (Gupta/Das), 
``software-defined networking'' (Sekar) and core graduate
courses in ``networking'' (Sekar/Das). The PIs regularly 
scope out suitable topics from their existing research projects
to assign term projects to these graduate level classes. Over the past
years, such projects motivated students better due to their
direct contribution to a bigger effort. This improved participation
as well as the final results, often students continuing working on these topics past the 
course term, writing papers or developing various
 artifacts of longer term value. 

%especially by having  relevant project topics and lab support
%available to the students. We also plan to develop tutorial materials 
%on data center networking and FSO communications, present such 
%tutorials in relevant conferences and finally make them available freely
%via YouTube. 

%\vyas{probably need some concrete pointers here on wireless classes,
%SDN/advanced classes, theory classes etc that we have taught and generated some
%tangible research from}
%
%\samir{addressed.}

\para{Engaging High School and Undergraduate Students.}  Long Island
have some of the best public schools in the country and we are keen on
tapping into this high school talent. SBU has a Simons Summer
Research Program\footnote{\normalsize
  http://www.stonybrook.edu/simons/} that provides a mechanism to
recruit talented high school students.  PIs Gupta and Das
have mentored students in the past in this program whenever
they had projects that high schooler could relate to easily (e.g.,
on RFID tracking and robot navigation). 
The proposed project brings in `hands-on' components
such as FSO  communications, use of steerable lasers to implement
network switching that we believe would be of interest to the
participating high schoolers in the Simons program. 
We hope to be able to recruit 1-2 such students each summer. 
% Students in the this program routinely competes
%nationally in the Intel Science
%Talent Search and often successfully with SBU professors as mentors. 
%
%During the past few summers at SBU, our colleagues have also organized
%Engineering Camps to attract high school students to come to SBU; At
%the camps, the students have several two-day laboratories in which
%they are instructed on how to design, build, and program various types
%of devices.
%%
%The above programs would provide perfect avenues to recruit a few
%high-school students for summer projects related to our research.


The PIs have strong history of involving undergraduates
in research and supporting them via REU supplements. This
approach motivated a few of them enough in the past
that they joined in the graduate program. The PIs
will continue this involvement for the proposed project.

%Due to the hi-tech appeal of free-space optics in data communications
%and other applications, we are keen on giving presentations and
%demonstrating appropriate aspects of our research prototype to some
%high-schools and our undergraduate students. We believe that the
%obvious appeal of free-space optics and steering mechanisms will be
%exciting for the students, and give us an opportunity to further
%encourage and recruit some of the best students.
%%
%Finally, we plan to build ``kits'' that can be used by the students to
%build hobby projects, e.g., FSO-based scanning devices, inexpensive
%custom-built steering mechanisms for FSO devices, demonstration of
%high-bandwidth FSO links using commodity hardware, etc. More elaborate
%projects based on the above ideas would be ideal for senior projects. 
%
%We hope to motivate them high-school students to pursue further
%education and careers in computer science. Many of the Simons Summer
%Research Program participants have excelled at the Intel Science
%Talent Competition (ISTC), and we are keen on mentoring high school
%students for ISTC.

%\vyas{is there concrete evidence .. this seems to say someone else in SB has done this, not necessarily us :) }
%
%\vyas{are the kits budgeted for?}

\para{Involving Under-Represented Groups.} 
The PIs have history of mentoring students
from the traditionally under-represented groups
in STEM disciplines. Gupta and Das has graduated
a woman PhD student each in the recent past and Das currently advises
one woman PhD student. Earlier, Das also graduated
a PhD student, Robert Casta\~{n}eda, from the Hispanic community and 
Casta\~{n}eda is currently 
a faculty member in a minority serving college in his native San Antonio. 
Das also served in several related NSF organized outreach efforts, e.g., mentoring faculty from minority-serving institutions
for writing competitive NSF grants. The PI will continue similar efforts
in connection with the proposed project. 

%SBU has a
%history of active outreach efforts in order to involve traditionally
%under-represented groups in science and engineering research. 
%This includes the Turner Fellowship Program minority
%for graduate students, the SUNY Alliance for Minority Participation (SUNY AMP), a
%minority faculty recruitment initiative, and the SUNY Alliance for
%Inclusive Graduate Education and the Professoriate (SUNY AGEP).
%Research in undergraduate studies will also be integrated through the
%\emph{Women In Science \& Engineering (WISE)} mentoring program in
%SBU, which regularly offers four-week research and inquiry-based
%courses. We plan to introduce a new WISE course related to free-space
%optics communications and applications. The PIs are committed to
%involve under-represented groups in ``high-tech'' research and
%development.

%\vyas{probably should say something abt pis track record in working
% with underrepresented gro?}

