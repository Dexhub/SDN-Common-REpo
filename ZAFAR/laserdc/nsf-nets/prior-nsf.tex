\section{Results From Prior NSF Support}

\textbf{Samir R. Das} and \textbf{Himanshu Gupta} are PI/Co-PIs on the following 
recently concluded/ongoing NSF awards:
%i) `A Measurement-Driven Physical-Interference-Based Approach for the Design of Mesh Networks' (2007-09, \$200,000), 
i) `A Market-Driven Approach to Dynamic Spectrum Sharing' (2008-13, \$406,000), and  ii) 
`Understanding Traffic Dynamics in Cellular Data Networks and Applications to Resource Management,' (2011-14, \$320,425).  
%The first award resulted in developing measurement-driven methodologies for understanding wireless 
%interference and exploiting them for transmission scheduling. 
%
These projects
focus on developing market-driven algorithms and systems for dynamic spectrum access systems (first)
and understanding spatio-temporal traffic dynamics in cellular data networks via analysis of network traces and using them
for spectrum/energy management applications (second). Over 15 papers were co-authored by the PIs 
related to these awards and 6 PhD students received
direct support. The PIs gave several public lectures based on the results.
%\red{[SD: if we have space we may also mention the sensor grants.]}. 
%and taught well-received 
%tutorials (on mesh networks and on interference modeling). 
%The awards
%also contributed to developing teaching materials in graduate and undergraduate wireless networking class
%that the PI teaches and generated
%or strengthened industry collaborations (NEC Research, Bell Labs and start-ups).

\textbf{Jon Longtin}, Co-PI, CBET-1048744, ``NSF/DOE Thermoelectrics Partnership:
Integrated Design and Manufacturing of Cost-Effective and
Industrial-Scalable TEG for Vehicle Applications,'' \$404,226,
10/1/10-09/30/13. This work developed thermoelectric materials and systems
for automotive applications using thermal spray technology. $\mathrm{Mg_{2}Si}$ and
filled Skudderudites were deposited directly onto exhaust system
components to convert exhaust waste heat to electricity. Key tasks on the
project included 1) materials development and characterization, 2)
short-pulse laser micromachining deposited thermoelectric materials, 3)
techniques for fabrication of multi-layer thermoelectric generators, and
4) device characterization and testing.  6 conference, 4 journal
publications, 4 Masters thesis and 1 Ph.D. thesis have resulted from the
project.


%% VS: not sure what to say here .. it has cloud in the title so i do want to say something to distance it and also say that i cant have anything to show since it just got funded
\textbf{Vyas Sekar} is a PI on two recently awarded NSF grants ``Enabling
Flexible Middlebox Processing in the Cloud'' and ``Rethinking Security in the
Era of Cloud Computing'' starting in Sep 2013. The research proposed therein
focuses largely on ``middlebox'' functionality such as IDS, firewall, and
proxies and does not focus on the datacenter topology and routing aspects.
These projects have just commenced and there are no outputs at this time.  As
such the proposed research in these projects does not overlap with the
management layer/SDN approaches proposed here. 

